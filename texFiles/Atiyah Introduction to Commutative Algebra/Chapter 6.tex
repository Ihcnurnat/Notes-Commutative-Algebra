\section{Chapter 6}

\subsection{Corollary 6.4}

One might wonder if $\bigoplus_{i\in I}M_i$ is also Noetherian given the index set $I$ is not necessarily finite with each $M_i$ being Noetherian. 
Consider this example 3) on Page 76
$$k[x_1,x_2,...]=\langle x_1\rangle \oplus\langle x_2\rangle\oplus\cdots.$$
Not sure if the above example works well as a \textit{counterexample}???

See a post \href{https://math.stackexchange.com/questions/173614/question-about-direct-sum-of-noetherian-modules-is-noetherian}{HERE}. 
Once we allow infinte index set, at least countably infinite, we'll get $M_1\subsetneq M_1\oplus M_2\subsetneq...$

\subsection{Proposition 6.7}

We cannot interpret this inclusion set-theorically? Any element $n_{0}+N_i$ must also be a.... I cannot resolve the coset $+N_i$ problem... 

The inclusion $N_{i-1}/N_i\subset M_{i-1}/M_i$ is given by noticing $N_i$ is \textit{exactly} kernel of the map (by definition the kernel is exactly $M_i\cap N=N_{i})$
% https://q.uiver.app/#q=WzAsMyxbMCwwLCJOX3tpLTF9Il0sWzEsMCwiTV97aS0xfSJdLFsyLDAsIk1fe2ktMX0vTV9pIl0sWzAsMSwiIiwwLHsic3R5bGUiOnsidGFpbCI6eyJuYW1lIjoiaG9vayIsInNpZGUiOiJ0b3AifX19XSxbMSwyLCJcXHBpIl1d
\[\begin{tikzcd}
	{N_{i-1}} & {M_{i-1}} & {M_{i-1}/M_i}
	\arrow[hook, from=1-1, to=1-2]
	\arrow["\pi", from=1-2, to=1-3]
\end{tikzcd}\]
And this induces an injective from $N_{i-1}/N_i\to M_{i-1}/M_i$, for which we can interpret it as set inclusion.

\subsection{Ex 1.}

\subsubsection{(i)}

Since $M$ is Noetherian, then it's finitely generated in particular. 
Then we can apply one form of Nakayama's Lemma.
One approach is to use induction on number of generators of $M$ ??? 

I didn't get the approach suggested by the hint \href{https://math.stackexchange.com/questions/145310/proving-that-surjective-endomorphisms-of-noetherian-modules-are-isomorphisms-and}{HERE}.
One crutial observation is the ascending chain of modules
$$\operatorname{Ker}(u)\subset\operatorname{Ker}(u^2)\subset\operatorname{Ker}(u^3)\subset\cdots$$

\subsubsection{(ii)}

Couldn't understand hint from both the textbook and the post \href{https://math.stackexchange.com/questions/145310/proving-that-surjective-endomorphisms-of-noetherian-modules-are-isomorphisms-and}{HERE}. 

A post \href{https://math.stackexchange.com/questions/273181/if-m-is-an-artinian-module-and-f-m-to-m-is-an-injective-homomorphism-then}{HERE}.

According to Artinian assumption there exists $n\in \mathbb Z$ such that 
$$\operatorname{Im}(u)\supset\operatorname{Im}(u^2)\supset\cdots\supset \operatorname{Im}(u^n)=\operatorname{Im}(u^{n+1})=\cdots$$

And we only need to prove the claim $M=\operatorname{Im}(u^n)+\operatorname{Ker}(u^n)$... 

Another cute argument is to consider any $m\in M$. There exists $m'\in M$ such that 
$$u^n(m)=u^{n+1}(m').$$
By injectivity of $u$ we must have $u^{n-1}(m)=u^n(m')$. Inductively, we know $m=u(m')$. Hence $\operatorname{Im}(u)=M$.

\subsubsection{}
For (ii), assumption of Artinian is necessary. Since for non-Artinian $\mathbb Z$-module $\mathbb Z$, we have $\times 2:\mathbb Z\to \mathbb Z$ injective but not surjective.