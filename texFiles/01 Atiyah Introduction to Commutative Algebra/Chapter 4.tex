%\section{Chapter 4}

\section{Example 3) Page 51}

My question was about that quotient ring... 

See a post \href{https://math.stackexchange.com/questions/93478/is-each-power-of-a-prime-ideal-a-primary-ideal}{HERE}. 

See a post dicussing the quotient ring $k[x,y,z]/\langle xy-z^2\rangle$ \href{https://math.stackexchange.com/questions/3320367/mathbbcx-y-z-xy-z2-is-not-a-field}{HERE}. 

\section{Prop 4.2}

For the original proof. Note that $A/\mathfrak a$ is a local ring, then elements in $A/\mathfrak a$ is either a unit or a nonunit. In case of a nonunit, it lies in the nilradical (intersection of all prime ideals).

Another proof is \href{https://math.stackexchange.com/questions/649146/an-ideal-whose-radical-is-maximal-is-primary}{HERE}.

Proof is based on a result from Exercise 10 Chapter 1 of \cite{atiyah1994introduction}.

\section{Lemma 4.3.}

Change "for some $i$" to "for any $i$" \todo{?}

Notice that radical commute with finite intersection of ideals, see \href{https://en.wikipedia.org/wiki/Radical_of_an_ideal}{HERE}. \todo{Verified.}
How about infinite intersection of ideals?

\section{Lemma 4.4.}

(i) Notice \[1\in (\mathfrak q:x)=\{y\in A ~\mid~ y(x)\subset \mathfrak q\}.\]

(ii) Clearly we have $\mathfrak q\subset (\mathfrak q:x)$.
Conversely, for $y\in (\mathfrak q:x)$, then $yx\in \mathfrak p$. While $x\notin \mathfrak p$, it follows that $y\in \mathfrak q$.

(iii) Clearly $(\mathfrak q : x)\supset \mathfrak p$. Conversely, for $y\in (\mathfrak q : x)$, we have $yx\in \mathfrak q$.

\section{Proposition 4.8.}

For more details, see Lemma 8.33. of Andreas Gathmann's notes on commutative algebra.

\section{Exercise 1.}
\textit{If an ideal $\mathfrak a$ has a primary decomposition, then $\Spec (A/\mathfrak a)$ has only finitely many irreducible components.
}
\begin{proof}
    According to Exercise 20 Chap 1 \cite{atiyah1994introduction}, irreducible components of $\Spec (A/\mathfrak a)$ are closed sets 
    \[\mathbf V(\mathfrak p)\] where $\mathfrak p\triangleleft A/\mathfrak a$ is a \textit{minimal prime ideal}. While ideal $\mathfrak a$ admits a primary decomposition, then 
    \[\mathfrak p\supset \mathfrak a=\mathfrak q_1\cap ...\cap \mathfrak q_n\] for primary ideals $\mathfrak q_i$ and $n\in\mathbb N$. Given $\mathfrak p$ is a prime ideal, then 
    \begin{align*}
        \mathfrak p \supset&~ \sqrt{\mathfrak a}\\
        =&~ \sqrt{\mathfrak q_1\cap ...\cap \mathfrak q_n}\\
        =&~ \sqrt{\mathfrak q_1}\cap ...\cap \sqrt{\mathfrak q_n} ~\text{ by Exercise 1.13. \cite{atiyah1994introduction}}.
    \end{align*}Apply Prime Avoidance Proposition 1.11. \cite{atiyah1994introduction}, we know $\mathfrak p\supset \sqrt{\mathfrak q_i}$ for some integer $1\leq i\leq n$. Notice that $\sqrt{\mathfrak q_i}$ is a prime ideal, then to be minimal prime ideal we must have $\mathfrak p=\sqrt{\mathfrak q_i}$. There are only finitely many prime ideals $\sqrt{\mathfrak p_i}$, each corresponds to a minimal prime ideal and then an irreducible components. 
    Hence $\Spec(A/\mathfrak a)$ can only have finitely many irreducible components.

\end{proof}

\subsection{Comments}

See this link regarding the converse of the statement \href{https://math.stackexchange.com/questions/207406/irreducible-components-of-the-prime-spectrum-of-a-quotient-ring-and-primary-deco}{HERE}.

\section{Exercise 2.}
\textit{If $\mathfrak a = \sqrt{\mathfrak a}$, then $\mathfrak a$ has no embedded prime ideals.}

    Assume $\mathfrak a$ admits a primary decomposition $\mathfrak a = \mathfrak q_1\cap...\cap\mathfrak q_n$.
\begin{proof}
    Notice that 
    \begin{align*}
        \sqrt{\mathfrak a}=\mathfrak a =&~ \bigcap \mathfrak q_{\bullet}=\bigcap_{1\leq i\leq n}\sqrt{q_i}\\
        \Rightarrow~ \bigcap_{\mathfrak p\supset \mathfrak a}\mathfrak p=&~\bigcap_{1\leq i\leq n}\sqrt{q_i}.
    \end{align*}
    For the set $A:=\{\sqrt{\mathfrak q_1},...,\sqrt{\mathfrak q_n}\}$, we need to show that each $\sqrt{\mathfrak q_i}$ is minimal with respect to inclusion. But in fact, we know 
    \[A\subset\{~ \mathfrak p\supset \mathfrak a,~\mathfrak p ~\text{is a prime ideal}~\}.\]
    The fact that when taking intersection, we have an equality implies that each element of the set $A$ is minimal, for otherwise we have 
    \[\bigcap_{\mathfrak p\supset \mathfrak a}\mathfrak p \subsetneqq \bigcap_{\mathfrak p\in A}\mathfrak p=\bigcap_{1\leq i\leq n}\sqrt{q_i}\]
\end{proof}

\section{Exercise 3.}
\textit{}

\begin{proof}
    
\end{proof}