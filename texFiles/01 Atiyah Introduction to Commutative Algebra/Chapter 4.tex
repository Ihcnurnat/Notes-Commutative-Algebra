\section{Chapter 4}

\subsection{Example 3) Page 51}

My question was about that quotient ring... 

See a post \href{https://math.stackexchange.com/questions/93478/is-each-power-of-a-prime-ideal-a-primary-ideal}{HERE}. 

See a post dicussing the quotient ring $k[x,y,z]/\langle xy-z^2\rangle$ \href{https://math.stackexchange.com/questions/3320367/mathbbcx-y-z-xy-z2-is-not-a-field}{HERE}. 

\subsection{Prop 4.2}

For the original proof. Note that $A/\mathfrak a$ is a local ring, then elements in $A/\mathfrak a$ is either a unit or a nonunit. In case of a nonunit, it lies in the nilradical (intersection of all prime ideals).

Another proof is \href{https://math.stackexchange.com/questions/649146/an-ideal-whose-radical-is-maximal-is-primary}{HERE}.

Proof is based on a result from Exercise 10 Chapter 1 of \cite{atiyah1994introduction}.

\subsection{Lemma 4.3.}

Change "for some $i$" to "for any $i$" \todo{?}

Notice that radical commute with finite intersection of ideals, see \href{https://en.wikipedia.org/wiki/Radical_of_an_ideal}{HERE}. \todo{Verified.}
How about infinite intersection of ideals?

\subsection{Lemma 4.4.}

(i) Notice \[1\in (\mathfrak q:x)=\{y\in A ~\mid~ y(x)\subset \mathfrak q\}.\]

(ii) Clearly we have $\mathfrak q\subset (\mathfrak q:x)$.
Conversely, for $y\in (\mathfrak q:x)$, then $yx\in \mathfrak p$. While $x\notin \mathfrak p$, it follows that $y\in \mathfrak q$.