%\section{Chapter 5}

\section{Proposition 5.1}

For iii) $\Rightarrow$ iv), notice that $C$ is a subring and $1\in C$ in particular, which helped us to prove the faithful property.

For iv) $\Rightarrow$ i), see Gong Ting's notes \href{https://drive.google.com/file/d/12aM32lvgwaq9DI_ydv93aIty7Q1MhRmw/view}{HERE} on Page 30, in which the last part used "determinant trick".

\section{Corollary 5.4.}

This is in fact iff. 

If we know ring extensions $A\subset B\subset C$. Assume $A\subset C$ is integral then $A\subset B$ is integral for $b\in B\subset C$; and $B\subset C$ is integral for $c\in C$ we could use exactly the same monic polynomial. 

The only nontrivial direction is to prove $A\subset B$, $B\subset C$ are integral, then the tower $A\subset C$ is integral. 

For a more complete description, see Gathmann's Notes Lemma 9.6. on Page 81. For the converse direction of the statement, namely if we know $A\subset C$ then ... It's very important $A, B$ are both rings. See Stacks Project: finite ring map! \todo{Checked!}


\section{Proposition 5.6.}

\subsection{(i)}
Here I present a re-write of the proof. 

% https://q.uiver.app/#q=WzAsMyxbMCwwLCJBIl0sWzEsMCwiQiJdLFsyLDAsIkIvXFxtYXRoZnJhayBiIl0sWzEsMiwiXFxwaV8yIl0sWzAsMSwiXFxpb3RhIiwwLHsic3R5bGUiOnsidGFpbCI6eyJuYW1lIjoiaG9vayIsInNpZGUiOiJ0b3AifX19XV0=
\[\begin{tikzcd}
	A & B & {B/\mathfrak b}
	\arrow["{\pi_2}", from=1-2, to=1-3]
	\arrow["\iota", hook, from=1-1, to=1-2]
\end{tikzcd}\]

Composition of the map has kernel $\operatorname{Ker}(\pi_2\circ\iota)= \mathfrak b^c$, which induces an injective (well-defined) map $f:A/\mathfrak a\to B/\mathfrak b$ so that the following diagram commute

% https://q.uiver.app/#q=WzAsNCxbMCwwLCJBIl0sWzEsMCwiQiJdLFsxLDEsIkIvXFxtYXRoZnJhayBiIl0sWzAsMSwiQS9cXG1hdGhmcmFrIGEiXSxbMCwzLCJcXHBpXzEiLDJdLFsxLDIsIlxccGlfMiIsMl0sWzAsMSwiXFxpb3RhIiwwLHsic3R5bGUiOnsidGFpbCI6eyJuYW1lIjoiaG9vayIsInNpZGUiOiJ0b3AifX19XSxbMywyLCJmIiwyLHsic3R5bGUiOnsiYm9keSI6eyJuYW1lIjoiZGFzaGVkIn19fV1d
\[\begin{tikzcd}
	A & B \\
	{A/\mathfrak a} & {B/\mathfrak b}
	\arrow["{\pi_1}"', from=1-1, to=2-1]
	\arrow["{\pi_2}"', from=1-2, to=2-2]
	\arrow["\iota", hook, from=1-1, to=1-2]
	\arrow["f"', dashed, from=2-1, to=2-2]
\end{tikzcd}\]

Suppose we have a monic polynomial such that for $x\in B$,
\[x^n+\iota(a_1)x^{n-1}+\cdots+\iota(a_n)=0\in B\] for $a_i\in A$ where integer $1\leq i\leq n$. By the commutativity of the diagram we have 
\begin{align*}
    \pi_2(0_A)=0=&\pi_2(x)^n+\pi_2\circ\iota(a_1)\pi_2(x)^{n-1}+\cdots+\pi_2\circ\iota(a_n)\\
    =&\pi_2(x)^n+f\circ\pi_1(a_1)\pi_2(x)^{n-1}+\cdots+f\circ\pi_1(a_n)
\end{align*}with all coefficients in $f(A/\mathfrak a)\subset B$. Notice that $\pi_2$ is surjective, this means for any element of $B/\mathfrak b$ we can find a monic polynomial in $f(A/\mathfrak a)[X]$ admits it as a solution. Therefore we know $B/\mathfrak b$ is integral over $A/\mathfrak a$.

Also see Gathmann's Notes Lemma 9.7. for extra information regarding inheritance of integral.

\subsection{Pathological}

Question: If we know integral extension of quotient rings $A/\mathfrak a\subset B/\mathfrak b$, do we know $A\subset B$ is integral? No. Consider a non-integral extension $R\subset R[x]$. Ideal $\langle x\rangle$ contract as $\langle x\rangle^c=0$, so 
\[R=R/(I\cap R)\subset R[x]/\langle x\rangle=R\] gives us an integral extension.

See my remarks on Gathmann's Notes Lemma 9.7.

\subsection{(ii)}

It's important that $S$ is un-changed! See an example where the implication will fail if localising with respect to different multiplicatively closed subset: Gathmann Commutative Algebra Exercise 9.8. (b).