%\section{Chapter 10}

\section{TOPOLOGIES AND COMPLETIONS}

\subsection{Linear Topology}

See a note about \href{https://math.gsu.edu/fenescu/commalglectures/8250Lect2.pdf}{$\mathfrak I$-adic topology}. See this \href{https://math.stackexchange.com/questions/4775023/linear-topology#:~:text=Definition%3A%20A%20linear%20topology%20τ,consisting%20of%20submodules%20of%20M.}{POST} and this note on \href{http://alpha.math.uga.edu/%7Epete/TopSection5.pdf}{NEIGHBORHOOD SUB/BASES}. For a quick introduction, see \href{https://stacks.math.columbia.edu/tag/07E7}{Tag 07E7} and \href{https://stacks.math.columbia.edu/tag/00M9}{Tag 00M9}. 

See notes on filtration, topologies, and completions \href{https://www.math.uchicago.edu/~may/MISC/Topologies.pdf}{HERE}. 

For fundamental neighborhood basis, see \href{https://math.stackexchange.com/questions/22957/topology-defined-by-a-fundamental-system-of-neighbourhoods-of-zero-in-a-topologi}{POST} and see \href{https://math.stackexchange.com/questions/69518/how-can-you-construct-a-topology-from-a-fundamental-system-of-neighborhoods?rq=1}{HERE} with \href{http://alpha.math.uga.edu/%7Epete/TopSection5.pdf}{NOTE}.

\subsection{Remarks on Page 101}

The reason why $G$ is Hausdorff. 
One equivalent condition for $G$ being Hausdorff is the diagonal element in $G\times G$ is closed. 
Because for point $(x,y)$ that's off-diagonal, i.e. $x\neq y$, we can always find $U\times V$ (by applying Hausdorff in $G$) that's open in $G\times G$ with trivial intersection to the diagonal. See a post \href{https://math.stackexchange.com/questions/136922/x-is-hausdorff-if-and-only-if-the-diagonal-of-x-times-x-is-closed}{HERE}.

\section{Lemma 10.1}

\subsection{(i)}
Solutions could be found \href{https://math.stackexchange.com/questions/13368/intersection-of-neighborhoods-of-0-subgroup}{HERE} and \href{https://math.stackexchange.com/questions/174955/intersection-of-all-neighborhoods-of-zero-is-a-subgroup}{HERE}. 
Also see \href{http://www.math.brown.edu/dabramov/MA/s1617/252/MA252Ch10.pdf}{HERE}.

\begin{proof}
	To check the \textit{set} $H$ is a group, it suffices to verify 
	\begin{enumerate}
		\item For $x\in H$, then $-x\in H$. For details see the second \href{https://math.stackexchange.com/questions/174955/intersection-of-all-neighborhoods-of-zero-is-a-subgroup}{POST} above.
		\item Closed under addtion.
	\end{enumerate}

\end{proof}

\subsection{Details of (ii)}
Details could be found at the same link \href{https://math.stackexchange.com/questions/174955/intersection-of-all-neighborhoods-of-zero-is-a-subgroup}{HERE}, the first answer.

\subsection{Details for (iii)}
For each element $g+H\in G/H$, it's the preimage of the continuous map
% https://q.uiver.app/#q=WzAsMixbMCwwLCJnK0giXSxbMSwwLCJIIl0sWzAsMSwiLWciXV0=
\[\begin{tikzcd}
	{g+H} & H
	\arrow["{-g}", from=1-1, to=1-2]
\end{tikzcd}\] While $H$ is closed, hence $g+H$ is closed as expected. In fact, this translation map is a homeomorphism, see Atiyah's comment before Lemma 10.1.

\subsection{Details for (iii), without incurring (ii)}
???potentially \textit{incorrect}!\\ Conversely, assume $H=0$. Pick distinct points $x_1,x_2\in G$, then we have $x_1-x_2\neq 0$ in $G$. 
While $H=\cap_{i\in I} U_i$ where each $U_i$ are all $0$-neighborhood. Hence we can find a $0$-neighborhood $U_1$ such that $$x_1-x_2\notin U ~\Rightarrow~ x_1\notin x_2+U_1.$$ 
Similarly, we can find another $U_2$ such that $$x_2-x_1\notin U ~\Rightarrow~ x_2\notin x_1+U_2.$$
Therefore $x_1+U_2\setminus U_1$ and $x_2+U_1\setminus U_2$ are two disjoint neighborhoods that contains $x_1,x_2$ respectively, which proves $G$ is Hausdorff.

\subsection{Details for (iv)}
$\Rightarrow$: Assume $G$ is Hausdorff but $H\neq 0$... Pick an element $y\in H$... See a post \href{URLhttps://math.stackexchange.com/questions/421109/questions-about-the-intersection-of-all-neighborhoods-of-0-in-a-topological-ab}{HERE}. 