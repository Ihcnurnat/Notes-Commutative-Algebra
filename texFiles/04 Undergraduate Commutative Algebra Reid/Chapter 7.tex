%\section{Chapter 7}

\section{7.1 Proposition}

For (v), see Atiyah \cite{atiyah1994introduction} Exercise 3 of Chapter 3 on taking localisation twice; see "A Term" \cite{altman} Proposition 11.16.

We get a ring map $(\cdot)_{P}:M_{Q}\to (M_Q)_P=M_P$, and if $M_Q=0$ then $M_P=0$. Taking contrapositive is what we're looking for.

\section{Definition 7.3}

In (ii), we have an equivalent definition. 
Assume $M$ contains a submodule isomorphic to $A/\mathfrak P$, i.e. 
% https://q.uiver.app/#q=WzAsMyxbMCwwLCJBL1xcbWF0aGZyYWt7UH0iXSxbMSwwLCJOIl0sWzIsMCwiTSJdLFsxLDIsIlxcc3Vic2V0ZXEiLDAseyJzdHlsZSI6eyJ0YWlsIjp7Im5hbWUiOiJob29rIiwic2lkZSI6InRvcCJ9fX1dLFswLDEsImY6XFxzaW1lcSJdXQ==
\[\begin{tikzcd}
	{A/\mathfrak{P}} & N & M
	\arrow["\subseteq", hook, from=1-2, to=1-3]
	\arrow["{f:\simeq}", from=1-1, to=1-2]
\end{tikzcd}\]
Now we take an element $x:=f(1+\mathfrak P)\in N\subset M$. Ring $R$ acts on $N$ as 
\todo{???}$$r\cdot m:= rf^{-1}(n)\in A/\mathfrak P.$$
Therefore we have $$\forall y\in\mathfrak P ~\Leftrightarrow~ yx=0\in A/\mathfrak P,$$which implies $\operatorname{Ann}(x)=\mathfrak P$.

Conversely, we use first isomorphism theorem of module as 
% https://q.uiver.app/#q=WzAsMyxbMCwwLCJBIl0sWzEsMCwiXFxsYW5nbGUgeFxccmFuZ2xlIl0sWzAsMSwiQS9cXG1hdGhmcmFrIFAiXSxbMCwyXSxbMCwxLCJcXGNkb3QgeCJdLFsyLDEsIlxcc2ltZXEiLDIseyJzdHlsZSI6eyJib2R5Ijp7Im5hbWUiOiJkYXNoZWQifX19XV0=
\[\begin{tikzcd}
	A & {\langle x\rangle} \\
	{A/\mathfrak P}
	\arrow[from=1-1, to=2-1]
	\arrow["{\cdot x}", from=1-1, to=1-2]
	\arrow["\simeq"', dashed, from=2-1, to=1-2]
\end{tikzcd}\]

\subsection{Remarks}
I'm not sure for the first part\dots

See "A Term" \cite{altman} Proposition 17.3 \todo{verified.}

\section{Proposition 7.4}

\subsection{(a)}
Element $y\in Ax$, which is a cyclic module, i.e. $\langle x\rangle\subset M$.
Clearly we have $\mathfrak P\subset \operatorname{Ann}(y)$ for any thing annihilates $x$ will also annihilate $y$. 
Conversely, we wish to show $\operatorname{Ann}(y)\subset \mathfrak P$. For any element $\alpha\in \operatorname{Ann}(y)\subset A$, we have $\alpha y=0$ for any nonzero $y\in Ax$. In proof of (a) it claim the isomorphism of rings between $\phi:Ax\to A/\mathfrak P$, which is an integral domain. Therefore we must have $\phi(\alpha)=0,$ which gives us $\operatorname{Ann}_A(\phi(y))\subset \mathfrak P$. One identity to note is \todo{??? Original proof is sketchy}
$$\operatorname{Ann}_A(y)=\operatorname{Ann}_A(\phi(y)) ~\Rightarrow~ \operatorname{Ann}_A(y)=\mathfrak P.$$

\subsection{(b)}

Uncanny tendency of "maximal" element to be prime ideal.

\subsection{(c)}

Direct usage of (b), and apply maximal element characterisation of Noetherian ring.

Is there any counterexample when $A$ isn't necessarily Noetherian?

\subsection{(d)}

This is where we could use definition that there exists an injective map $A/\mathfrak p\to M$.

\section{Proposition 7.8}

Here's another presentation of the original proof, with a different order.

\begin{proof}
$\Rightarrow$: Note that since $Q$ is $P$-primary, ideal $P$ is prime and therefore $Q\subset P\subsetneq A$. Then $A$-module $A/Q\neq 0$. While $A$ is Noetherian, apply Proposition 7.4. (c) gives us \[\text{Ass}~(A/Q)\neq \emptyset.\] 
By definition this implies there exists a non-zero element $x\in A/Q$ such that 
\[\text{Ann}~ x =\mathfrak q\in\text{Spec}A.\]

Now we can apply the original proof, which yield that $\text{Ann}~ x=P$ and it's a prime. 

$\Leftarrow$: The set of minimal primes $P'\supset \text{Ann}~ x$ is the same as set of minimal elements in $\text{Supp}~ M$. This is because $A/Q$ as an $A$-module is f.g., so is the subring $M\subset A/Q$.\todo{?} So we can apply Proposition 7.1 (iv) on Page 96 of \cite{reid1995undergraduate}.

\end{proof}