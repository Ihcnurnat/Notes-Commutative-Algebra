\section{Chapter 7}

\subsection{Definition 7.3}

In (ii), we have an equivalent definition. 
Assume $M$ contains a submodule isomorphic to $A/\mathfrak P$, i.e. 
% https://q.uiver.app/#q=WzAsMyxbMCwwLCJBL1xcbWF0aGZyYWt7UH0iXSxbMSwwLCJOIl0sWzIsMCwiTSJdLFsxLDIsIlxcc3Vic2V0ZXEiLDAseyJzdHlsZSI6eyJ0YWlsIjp7Im5hbWUiOiJob29rIiwic2lkZSI6InRvcCJ9fX1dLFswLDEsImY6XFxzaW1lcSJdXQ==
\[\begin{tikzcd}
	{A/\mathfrak{P}} & N & M
	\arrow["\subseteq", hook, from=1-2, to=1-3]
	\arrow["{f:\simeq}", from=1-1, to=1-2]
\end{tikzcd}\]
Now we take an element $x:=f(1+\mathfrak P)\in N\subset M$. Ring $R$ acts on $N$ as 
\todo{???}$$r\cdot m:= rf^{-1}(n)\in A/\mathfrak P.$$
Therefore we have $$\forall y\in\mathfrak P ~\Leftrightarrow~ yx=0\in A/\mathfrak P,$$which implies $\operatorname{Ann}(x)=\mathfrak P$.

Conversely, we use first isomorphism theorem of module as 
% https://q.uiver.app/#q=WzAsMyxbMCwwLCJBIl0sWzEsMCwiXFxsYW5nbGUgeFxccmFuZ2xlIl0sWzAsMSwiQS9cXG1hdGhmcmFrIFAiXSxbMCwyXSxbMCwxLCJcXGNkb3QgeCJdLFsyLDEsIlxcc2ltZXEiLDIseyJzdHlsZSI6eyJib2R5Ijp7Im5hbWUiOiJkYXNoZWQifX19XV0=
\[\begin{tikzcd}
	A & {\langle x\rangle} \\
	{A/\mathfrak P}
	\arrow[from=1-1, to=2-1]
	\arrow["{\cdot x}", from=1-1, to=1-2]
	\arrow["\simeq"', dashed, from=2-1, to=1-2]
\end{tikzcd}\]

\subsubsection{}
I'm not sure for the first part\dots

\subsection{Proposition 7.4}

\subsubsection{Proof (a)}
Element $y\in Ax$, which is a cyclic module, i.e. $\langle x\rangle\subset M$.
Clearly we have $\mathfrak P\subset \operatorname{Ann}(y)$ for any thing annihilates $x$ will also annihilate $y$. 
Conversely, we wish to show $\operatorname{Ann}(y)\subset \mathfrak P$. For any element $\alpha\in \operatorname{Ann}(y)\subset A$, we have $\alpha y=0$ for any nonzero $y\in Ax$. In proof of (a) it claim the isomorphism of rings between $\phi:Ax\to A/\mathfrak P$, which is an integral domain. Therefore we must have $\phi(\alpha)=0,$ which gives us $\operatorname{Ann}_A(\phi(y))\subset \mathfrak P$. One identity to note is \todo{??? Original proof is sketchy}
$$\operatorname{Ann}_A(y)=\operatorname{Ann}_A(\phi(y)) ~\Rightarrow~ \operatorname{Ann}_A(y)=\mathfrak P.$$