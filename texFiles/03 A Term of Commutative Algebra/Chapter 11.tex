%\section{Chapter 11}

\section{Exercise 11.5}

% https://q.uiver.app/#q=WzAsMyxbMCwwLCJSJ1xcdGltZXMgUicnIl0sWzEsMCwiUidcXHRpbWVzIFInJy9SJyciXSxbMSwxLCJCIl0sWzAsMV0sWzAsMiwiXFxwc2kiLDJdLFsxLDIsIiIsMCx7InN0eWxlIjp7ImJvZHkiOnsibmFtZSI6ImRhc2hlZCJ9fX1dXQ==
\[\begin{tikzcd}
	{R'\times R''} & {R'\times R''/R''} \\
	& B
	\arrow[from=1-1, to=1-2]
	\arrow["\psi"', from=1-1, to=2-2]
	\arrow[dashed, from=1-2, to=2-2]
\end{tikzcd}\]

Notice that we have $R''\subset \operatorname{Ker}\psi$, then $\psi$ descents to a unique map on the quotient from $R'\times R''/R''\to B$. And this map serves to prove the UMP of $R'\times R''/R''=R'$ as $S^{-1}R$. Hence we have $R'\sim S^{-1}R$. But is this the equality set-theoretic?

\section{Corollay (11.4).}

Notice that isomorphism will preserve units.

% https://q.uiver.app/#q=WzAsMyxbMCwwLCJSIl0sWzEsMCwiU157LTF9UiJdLFsxLDEsIlIiXSxbMCwxLCJcXHZhcnBoaV9TIl0sWzAsMiwiXFxvcGVyYXRvcm5hbWV7aWR9X1IiLDJdLFsxLDIsIlxccmhvIiwwLHsic3R5bGUiOnsiYm9keSI6eyJuYW1lIjoiZGFzaGVkIn19fV1d
\[\begin{tikzcd}
	R & {S^{-1}R} \\
	& R
	\arrow["{\varphi_S}", from=1-1, to=1-2]
	\arrow["{\operatorname{id}_R}"', from=1-1, to=2-2]
	\arrow["\rho", dashed, from=1-2, to=2-2]
\end{tikzcd}\]

UMP tells us there exists a unique map $\rho:S^{-1}R\to R$ such that $\operatorname{id}_R=\rho\varphi_S$. Notice that the inverse of $\varphi_S$ will also satisfy this requirement, by the uniquess we know it must be $\rho$.


\section{11.25 Exercise}

\begin{proof}
Here I interpret $R_{\lambda}$ as for some index $\lambda$ in certain index set, not as a localisation.\todo{?}

Homomorphic image of multiplicatively closed subset is again multiplicatively closed. And union of multiplicatively closed subset of a ring is again multiplicatively closed. Therefore $S$ defined in prompt is multiplicatively closed. 

Now we fix a morphism from $R\to A$ where $A$ is arbitrary ring. This data is equivalent to a co-cone over $A$. And we can build... \todo{See solution, it's much cleaner}

\end{proof}

\subsection{Another Remark}
See nLab \href{https://ncatlab.org/nlab/show/adjoints+preserve+%28co-%29limits}{HERE}. And a \href{https://math.stackexchange.com/questions/505479/does-localization-commute-with-direct-inverse-limits}{post}, and \href{https://math.stackexchange.com/questions/1902377/localization-as-colimit-defining-the-preorder}{post}. It could be easier to think about it as following. But there's a slight problem I couldn't fix: when commuting lim and tensor I couldn't figure out which is the correct multiplicative closed to localise.
\begin{proof}
	We could interpret localisation as a tensor product, which is a left adjoint. Hence it must commute with taking colimit.
\end{proof}
