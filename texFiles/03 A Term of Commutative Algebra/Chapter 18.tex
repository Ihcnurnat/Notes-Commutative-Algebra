\section{Chapter 18}

\subsection{Definition (18.1).}

Notice that the definition is for a module $Q$ to be $\mathfrak p$-primary in module $M$.

This is very general. When we set $Q\triangleleft M=R$ as an ideal, and if we furthur assume ring $R$ is Noetherian we'll have \[Q ~\text{is}~ \mathfrak p-primary ~\Leftrightarrow~ \text{Ass}~(M/Q)=\{\mathfrak p\}.\]

This is precisely the statement of Theorem 7.8. on Page 105 of \cite{reid1995undergraduate}.

Therefore it's indeed a generalisation of the definition of primary of modules based on the rings.