This is for Gathmann's notes \href{https://agag-gathmann.math.rptu.de/en/commalg.php}{HERE}.

\section{Exercise 8.8.}

\subsection{(i)}
For $\mathbb R[x,y]/(y-x^2)$, we know it's $R[x]$ hence UFD.
\subsection{(ii)}
For $\mathbb R[x,y]/(xy-1)$, we know 
\[\mathbb R[x,y]/(xy-1)\sim k[y]_y=k[y,y^{-1}].\]

\textit{Localisation of UFD is UFD}
See a post \href{https://math.stackexchange.com/questions/2066749/localization-preserves-ufds-using-kaplansky-criterion}{HERE}: Localisation preservees UFD, using kaplansky's Criterion.
See Ben's notes on UFD and kaplansky's Criterion \href{https://public.websites.umich.edu/~brgould/disc4.pdf}{HERE}; Pete Clark' notes on Page 237 \href{http://alpha.math.uga.edu/%7Epete/integral.pdf}{HERE}; and a post \href{https://math.stackexchange.com/questions/140584/about-the-localization-of-a-ufd/140596#140596}{HERE}.

One fact to note is that being an integral domain isn't a local property. $R$ might not be an integral domain although the localisations $R_{\mathfrak p}$ are integral domains for all maximal ideals $\mathfrak p\trianglelefteq R$.


\subsection{(iii)}
For $\mathbb C[x,y]/(x^2+y^2-1)$, we have the isomorphism according a post \href[]{https://math.stackexchange.com/questions/1311675/show-that-mathbbcx-y-x2y2-1-is-a-ufd}{HERE} that it's in fact a Laurent 
\[\mathbb C[x,y]/(x^2+y^2-1)\sim \mathbb C[e^{it},~e^{-it}]\]

See a post \href{https://math.stackexchange.com/questions/244460/ring-of-trigonometric-functions-with-real-coefficients}{HERE}.

One thing to notice is that $\mathbb R[x,y]/(x^2+y^2-1)$ is not UFD but $\mathbb C[x,y]/(x^2+y^2-1)$ is PID.

\section{Lemma 9.6.}

\begin{comment}
Both statement (i) and (ii) are in fact iff.
Assume $R\subset R''$ are finite, i.e. \[R''=R\langle s\in S\rangle\] for a finite subset $S\subset R''$ of generators. Every element $x\in R'\subset R''$ could be express as a $R$-linear (coefficients in $R$) combination with generators in $S$. Therefore $R\subset R'$ is finite. For $R'\subset R''$, pick $x\in R''$ with expression and consider the coefficient of $R$ in $R'$. Generating set $S$ won't change, hence $R''$ could be regarded as a f.g. $R'$-module.
\end{comment}

Partial converse of (a): Assume $R\subset R''$ is finite. 

\subsection{Remarks}

See a post \href{https://math.stackexchange.com/questions/43353/transitivity-of-finitely-generated-condition}{HERE} discussing various transitivity results. 

For more interesting discussion regarding various counterexmaples, see Atiyah \cite{atiyah1994introduction} Corollary 5.4. \Cref{Atiyah Corollary 5.4.}.

\section{Lemma 9.7.}

\subsection{Counterexamples: Two Finite Towers}

Question on a partial "converse" of (a): If we know $R/(I\cap R)\subset R'$ is integral, do we know $R\subset R'$ is integral? No. See remarks on Proposition 5.6. of Atiyah's \cite{atiyah1994introduction}.

\section{Exercise 9.8.}

\subsection{(a)}

No.

Let $\beta=1/2\sqrt[3]{3}$, I claim it's not integral over $\mathbb Z$ for any polynomial $f\in \mathbb Z[x]$ admits $\beta$ as a root, look at the denominator will lead a contradiction.\todo{?}

And then apply Prop 9.5., which is "iff"!

\subsubsection{Hint}

See a post \href{https://math.stackexchange.com/questions/3770156/sqrt2-sqrt2-frac12-sqrt33-is-not-integral-over-mathbbz-s}{HERE}, \href{https://math.stackexchange.com/questions/2187396/checking-whether-a-given-element-is-integral-over-mathbbz}{HERE}.