This is for Gathmann's notes \href{https://agag-gathmann.math.rptu.de/en/commalg.php}{HERE}.

\section{Exercise 8.8.}

\subsection{(i)}
For $\mathbb R[x,y]/(y-x^2)$, we know it's $R[x]$ hence UFD.
\subsection{(ii)}
For $\mathbb R[x,y]/(xy-1)$, we know 
\[\mathbb R[x,y]/(xy-1)\sim k[y]_y=k[y,y^{-1}].\]

\textit{Localisation of UFD is UFD}
See a post \href{https://math.stackexchange.com/questions/2066749/localization-preserves-ufds-using-kaplansky-criterion}{HERE}: Localisation preservees UFD, using kaplansky's Criterion.
See Ben's notes on UFD and kaplansky's Criterion \href{https://public.websites.umich.edu/~brgould/disc4.pdf}{HERE}; Pete Clark' notes on Page 237 \href{http://alpha.math.uga.edu/%7Epete/integral.pdf}{HERE}; and a post \href{https://math.stackexchange.com/questions/140584/about-the-localization-of-a-ufd/140596#140596}{HERE}.

One fact to note is that being an integral domain isn't a local property. $R$ might not be an integral domain although the localisations $R_{\mathfrak p}$ are integral domains for all maximal ideals $\mathfrak p\trianglelefteq R$.


\subsection{(iii)}
For $\mathbb C[x,y]/(x^2+y^2-1)$, we have the isomorphism according a post \href[]{https://math.stackexchange.com/questions/1311675/show-that-mathbbcx-y-x2y2-1-is-a-ufd}{HERE} that it's in fact a Laurent 
\[\mathbb C[x,y]/(x^2+y^2-1)\sim \mathbb C[e^{it},~e^{-it}]\]

See a post \href{https://math.stackexchange.com/questions/244460/ring-of-trigonometric-functions-with-real-coefficients}{HERE}.

One thing to notice is that $\mathbb R[x,y]/(x^2+y^2-1)$ is not UFD but $\mathbb C[x,y]/(x^2+y^2-1)$ is PID.